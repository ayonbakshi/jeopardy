\documentclass{article}

\begin{document}

\begin{center}
  \huge{User Guide}
\end{center}

\section{Getting Started}
\label{sec:getting-started}

Welcome to the magnificent game of Jeopardy! To start a new game,
please enter the names of all three players and press start. If you
are unsure of how to play, please refer to the instructions, which can
be accessed from the starting game panel.

\section{Brief Instructions}
\label{sec:brief-instructions}

Once in the game board screen, Player 1 will pick a question. The
player will attempt to answer the question. If they are wrong, it is
the next player’s (1 $\rightarrow$ 2, 2 $\rightarrow$ 3, 3
$\rightarrow$ 1) turn to answer the question. If all three players get
it wrong, the correct answer will be revealed and the question will no
longer be available for picking. The correct answerer will be awarded
with an amount of money (indicated on the question button) and will be
able to select the next question. There is a special double jeopardy
question that is assigned by random. If a player selects this
question, they will get a chance to wager an amount (from \$5 to
either their current score or \$1000, whichever is higher) and answer
the question. They are the only player that can answer (players don’t
take turns answering if the previous player was wrong). If they are
correct, they win the amount of money they wagered, and they lose if
they are wrong. When all the questions are answered, the player with
the highest score wins, and the game is over.

\section{Features}
\label{sec:features}

\begin{itemize}
\item Instructions on starting panel for those who don't know who to play
\item Custom button graphics specific to \texttt{Question} objects
\item Dynamic frame/component resizing: finds the size of the screen
  and makes the frame a large as possible while maintaining a 4:3
  aspect ratio
\item Daily doubles randomly on one question
\item Questions that are buttons: the class holds question data but
  acts like a button in the GUI
\item Several classes that extend \texttt{JPanel}
  \begin{itemize}
  \item Create better organization
  \item Have some code that is unique to each object
  \end{itemize}
\item \texttt{GameUtils} class
  \begin{itemize}
  \item Utilities used in other classes
  \item Reduces code duplication
  \end{itemize}
\item Reads questions from a file
  \begin{itemize}
  \item File in an easy-to-extend CSV format
  \item Uses a \texttt{Scanner}, used to create \texttt{Question}
    objects
  \end{itemize}
\item Flexible file locations
  \begin{itemize}
  \item All files that the code can read are in one folder
  \item The code that finds searches more than one place and can be
    easily extended to search more
  \end{itemize}
\item Dialogues
  \begin{itemize}
  \item Used when there is an error retrieving a file
  \item Used for the instructions
  \item Used for daily double wager
  \end{itemize}
\end{itemize}

\section{Limitations}
\label{sec:limitations}

\begin{itemize}
\item Limited number of topics - any given topic will likely come up
  two games in a row
\item Only one round, TV Jeopardy! games have 3
\item The code that parses the question file will break if it has errors
\end{itemize}

\section{Bugs}
\label{sec:bugs}

\begin{itemize}
\item Buttons for answers sometimes change sizes when the player
  answers a question incorrectly
\end{itemize}

\end{document}